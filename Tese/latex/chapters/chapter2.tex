\documentclass[a4paper,12pt]{report}
\usepackage[utf8]{inputenc}
\usepackage[T1]{fontenc}
\usepackage[portuguese]{babel}
\usepackage{hyperref}
\usepackage{pifont}
\usepackage[table]{xcolor}
\usepackage{caption}
\usepackage{float}
\usepackage{graphicx}

\newcommand{\cmark}{\ding{51}} % ✓

\captionsetup[table]{skip=10pt}
\renewcommand{\arraystretch}{1.3}

\begin{document}

\chapter{Estado da Arte}

\section{Tecnologias Digitais nos Museus}

\textbf{História:} As tecnologias digitais introduziram profundas transformações em toda a cadeia de valor dos museus, desde a catalogação e preservação digital até à experiência de visita presencial e remota. Na década de 1990, a digitalização de acervos permitiu maior organização interna, mas foi com a massificação da Internet, dos dispositivos móveis e do acesso à realidade aumentada e virtual, no final dos anos 2000 e 2010, que se iniciou uma nova era de inovação no contacto com os públicos. Os museus portugueses, acompanhando a tendência global, lançaram plataformas online, campanhas de digitalização massiva, catálogos acessíveis remotamente e, gradualmente, aplicações móveis para visita enriquecida, propondo narrativas interativas e inclusão de conteúdos multimedia direcionados a diferentes faixas etárias e perfis de visitante. O acesso à Google Arts \& Culture, o investimento em apps próprias e em projetos de realidade mista demonstram o papel assumido pelas tecnologias como eixos estratégicos de mediação e aproximação às comunidades, tornando o museu mais aberto, experimental e dinâmico.


\paragraph{}\textbf{Vantagens:}
\begin{itemize}
    \item Democratização do acesso: ferramentas digitais tornaram acessíveis milhares de peças e conteúdos a públicos geograficamente afastados, portadores de deficiência e diferentes gerações.
    \item Maior atratividade e autonomia dos visitantes: as soluções tecnológicas possibilitam visitas personalizadas, trilhos temáticos, experiências educativas em família ou em contexto escolar, bem como oportunidades de experimentação e exploração lúdica ou pedagógica.
    \item Otimização da gestão e da avaliação: a recolha automática de dados permite mapear os percursos e interesses do público, facilitando melhorias continuadas na programação e experiência museológica.
\end{itemize}
\paragraph{}\textbf{Desvantagens:}
\begin{itemize}
    \item Elevados custos tecnológicos e dependência de atualizações constantes, incluindo riscos de obsolescência precoce dos equipamentos ou plataformas.
    \item Barreiras para visitantes com reduzida literacia digital, idosos ou grupos sociais sem fácil acesso a dispositivos e Internet.
    \item Risco de sobreposição do digital ao património físico, quando a tecnologia não é articulada com objetivos museológicos claros, levando a experiências descontextualizadas ou superficiais.
\end{itemize}

Neste quadro sintetiza-se uma comparação dos jogos digitais mais relevantes que serão descritos nos subtópicos posteriores.

\begin{table}[H]
\centering
\resizebox{\textwidth}{!}{
\begin{tabular}{|l|c|c|c|c|c|}
\hline
\textbf{Jogo} & \textbf{Narrativa} & \textbf{Decisão / Impacto} & \textbf{Feedback Imediato} & \textbf{Alinhamento Curricular} & \textbf{Colaboração} \\
\hline
EcoAtitudes            & \cmark & \cmark & \cmark & \cmark &  \\
\hline
Game in the Museum     & \cmark & \cmark & \cmark & \cmark & \cmark \\
\hline
Missão Arqueólogo VR   & \cmark & \cmark & \cmark & \cmark &  \\
\hline
ARTe Quiz              &  &  & \cmark & \cmark &  \\
\hline
\end{tabular}
}
\caption{Tabelas síntese dos jogos digitais museológicos mais relevantes}
\end{table}

% ---
\newpage
\subsection{Realidade Aumentada (AR)}

\textbf{História:} A Realidade Aumentada (AR) nos museus apresenta uma evolução marcada pela convergência entre a massificação dos smartphones e a busca por experiências educativas mais imersivas. Nos primeiros anos da década de 2010, grandes museus europeus e americanos lançaram pilotos de AR integrando animações sobre exposições e visitas temáticas, facilitando o cruzamento entre o real e o virtual em tempo real. Em Portugal, projetos como o ARTe Quiz e apps institucionais abriram caminho para a co-criação de conteúdos educativos entre equipas museológicas, tecnológicos e escolares. A implementação gradual da AR permitiu, por exemplo, visualizar objetos recriados em 3D, percursos gamificados e layers de informação histórica, promovendo o envolvimento e a aprendizagem ativa do público escolar e familiar, muitas vezes em parceria com entidades do ensino básico e secundário.

\paragraph{}\textbf{Vantagens:}
\begin{itemize}
    \item Potencia visitas interativas, promovendo autonomia, curiosidade e personalização do percurso.
    \item Permite apresentar conteúdos adaptados ao perfil do visitante, língua, faixa etária ou necessidade educativa especial.
    \item Estimula o envolvimento de públicos normalmente menos motivados por exposições tradicionais.
\end{itemize}
\paragraph{}\textbf{Desvantagens:}
\begin{itemize}
    \item Depende de equipamentos dos visitantes (telemóveis, tablets) e pode excluir quem não tem acesso.
    \item Exige investimento local em desenvolvimento, atualizações contínuas e validação pedagógica dos conteúdos.
    \item Pode provocar sobrecarga sensorial ou desvio de atenção das peças físicas, se não houver integração adequada com a mediação presencial.
\end{itemize}

\begin{table}[H]
\centering
\resizebox{\textwidth}{!}{
\begin{tabular}{|l|c|c|c|c|c|}
\hline
\textbf{Projeto/Aplicação} & \textbf{Guias AR} & \textbf{Jogos Integrados} & \textbf{Percurso Personalizado} & \textbf{Inclusão e Acessibilidade} & \textbf{Exploração Livre} \\
\hline
Museu App AR       & \cmark & \cmark & \cmark & \cmark &  \\
\hline
ARTe Quiz          & \cmark & \cmark & \cmark & \cmark &  \\
\hline
EcoAtitudes AR     &  & \cmark &  & \cmark &  \\
\hline
\end{tabular}
}
\caption{Exemplos e aspetos diferenciadores de implementações AR em museus}
\end{table}

% ---

\subsection{Realidade Virtual (VR)}

\textbf{História:} A Realidade Virtual em museus emergiu a partir de experiências pioneiras nos anos 2010, propondo visitas imersivas a espaços e acervos inacessíveis, ou simulando contextos históricos e científicos. No início, exigia hardware dedicado e era utilizada apenas em grandes eventos e projetos pontuais; com a democratização dos óculos VR e plataformas 3D, passou a integrar exposições fixas, visitas educativas, oficinas e até projetos de colaboração internacional envolvendo museus portugueses e estrangeiros. O desenvolvimento acelerado de conteúdos de reconstrução arqueológica, ciência e ambiente, associado à valorização da experimentação digital, colocou a VR como ferramenta estratégica de mediação e valorização do património.

\paragraph{}\textbf{Vantagens:}
\begin{itemize}
    \item Oferece experiências sensoriais imersivas, permitindo reconstituir espaços desaparecidos ou contextos históricos de difícil acesso.
    \item Facilita a experimentação, aprendizagem ativa, simulações e visitas remotas, sendo útil para públicos com mobilidade reduzida.
    \item Potencia o ensino interdisciplinar, cruzando história, ciência, geografia e literatura de forma envolvente.
\end{itemize}
\paragraph{}\textbf{Desvantagens:}
\begin{itemize}
    \item Investimento elevado em hardware, licenças e formação técnica das equipas.
    \item Limitações físicas (motion sickness, tempo de uso, adaptação) e barreiras a públicos com necessidades especiais não contempladas tecnicamente.
    \item Pode transformar-se em experiência isolada do património real, perdendo valor formativo se não articulada com atividades presenciais e mediadores.
\end{itemize}

\begin{table}[H]
\centering
\resizebox{\textwidth}{!}{
\begin{tabular}{|l|c|c|c|c|c|}
\hline
\textbf{Projeto} & \textbf{Visita 360º} & \textbf{Reconstrução 3D} & \textbf{Jogos VR} & \textbf{Foco Educativo} & \textbf{Livre vs Guiada} \\
\hline
Museu Virtual 360     & \cmark & \cmark &  & \cmark & \cmark \\
\hline
Design Museum VR      & \cmark & \cmark & \cmark & \cmark & \cmark \\
\hline
VISITOR Museums       & \cmark & \cmark & \cmark &  & \cmark \\
\hline
Missão Arqueólogo VR  &  & \cmark & \cmark & \cmark & \cmark \\
\hline
\end{tabular}
}
\caption{Exemplos e aspetos diferenciadores de implementações VR em museus}
\end{table}

% ---

\subsection{Jogos Digitais}

\textbf{História:} A apropriação dos jogos digitais por museus ocorreu inicialmente em plataformas web, CD-ROM e, posteriormente, em aplicações móveis e tablets. O crescimento do movimento serious games desde 2010 levou à criação de jogos de exploração, aventura e simulação ligados ao percurso expositivo, experimentados tanto em visitas presenciais como em virtual tours. Museus portugueses apostam em jogos que promovem o pensamento científico-crítico, a sensibilização ambiental e o trabalho em equipa, reforçando a ligação entre o visitante e as narrativas museológicas.

\paragraph{}\textbf{Vantagens:}
\begin{itemize}
    \item Elevam o envolvimento do visitante, promovendo sentimento de conquista, descoberta e resolução de desafios.
    \item Permitem feedback imediato, adaptando dificuldade e sugestões ao progresso do utilizador/aluno.
    \item Viabilizam a aprendizagem personalizada e a avaliação formativa, essenciais na integração curricular formal.
\end{itemize}
\paragraph{}\textbf{Desvantagens:}
\begin{itemize}
    \item Risco de dispersar o foco educativo se o equilíbrio entre diversão e aprendizagem não for garantido.
    \item Exigem recursos técnicos e pedagógicos para desenvolvimento, implementação e atualização.
    \item A acessibilidade pode não ser universal e nem todos os públicos demonstram predisposição para modelos lúdico-digitais.
\end{itemize}

\begin{table}[H]
\centering
\resizebox{\textwidth}{!}{
\begin{tabular}{|l|c|c|c|c|c|}
\hline
\textbf{Jogo} & \textbf{Narrativa} & \textbf{Decisão / Impacto} & \textbf{Feedback Imediato} & \textbf{Alinhamento Curricular} & \textbf{Colaboração} \\
\hline
EcoAtitudes            & \cmark & \cmark & \cmark & \cmark &  \\
\hline
Game in the Museum     & \cmark & \cmark & \cmark & \cmark & \cmark \\
\hline
ARTe Quiz              &  &  & \cmark & \cmark &  \\
\hline
Missão Arqueólogo VR   & \cmark & \cmark & \cmark & \cmark &  \\
\hline
Modelos 3D            &  &  & \cmark & \cmark & \cmark \\
\hline
\end{tabular}
}
\caption{Comparação de jogos digitais em museus: principais características}
\end{table}

% ---

\section{Jogos como Ferramentas de Comunicação e Aprendizagem}

\textbf{História:} O game-based learning (GBL) em museus evoluiu na última década, inspirado pelo sucesso das metodologias ativas no ensino formal. Os museus passaram a adotar jogos digitais, quizzes, simulações e missões como estratégias de mediação integradas no percurso expositivo. Estes recursos são idealizados em articulação com currículos escolares e educativos, fomentando autonomia, criatividade e colaboração dos visitantes.

\paragraph{}\textbf{Vantagens:}
\begin{itemize}
    \item Estimula autonomia, resolução criativa de problemas, pensamento crítico e colaboração.
    \item Permite avaliação contínua, feedback formativo imediato e motivação sustentada.
    \item Facilita a adaptação ao ritmo, interesses e estilo de aprendizagem de cada visitante ou grupo.
\end{itemize}
\paragraph{}\textbf{Desvantagens:}
\begin{itemize}
    \item Requer investimento pedagógico e tecnológico e parcerias entre equipa museológica e escolas/comunidade.
    \item Exige planeamento curricular rigoroso e atualização constante dos conteúdos.
    \item Pode gerar resistências em públicos habituados a mediação tradicional e não garantir acessibilidade digital plena.
\end{itemize}

\begin{table}[H]
\centering
\resizebox{\textwidth}{!}{
\begin{tabular}{|l|c|c|c|c|c|c|c|}
\hline
\textbf{Jogo/Atividade} & \textbf{Tipo} & \textbf{Narrativa} & \textbf{Decisão} & \textbf{Feedback} & \textbf{Colaboração} & \textbf{Currículo} & \textbf{Avaliação} \\
\hline
Quiz AR Exposições      & Quiz/AR   &  &  & \cmark &  & \cmark &  \\
\hline
Missão Arqueólogo VR    & Simulação & \cmark & \cmark & \cmark &  & \cmark & \cmark \\
\hline
EcoAtitudes             & Sério     & \cmark & \cmark & \cmark &  & \cmark & \cmark \\
\hline
Game in the Museum      & Aventura  & \cmark & \cmark & \cmark & \cmark & \cmark & \cmark \\
\hline
Modelos 3D              & Criativo  &  &  & \cmark & \cmark & \cmark &  \\
\hline
\end{tabular}
}
\caption{Comparação de exemplos de game-based learning em museus}
\end{table}

\end{document}
